\documentclass{article}
\usepackage{amsmath, amssymb}
\usepackage{graphicx}
\usepackage{hyperref}

\title{Mathematical Foundations of Bulletproofs in Rated Voting Systems}
\author{}
\date{}

\begin{document}

\maketitle

\section{Introduction}
This document explains the mathematical foundations behind the zero-knowledge proof protocol implemented using Bulletproofs and Pedersen commitments in a rated voting system. The focus is on how homomorphic properties of Pedersen commitments enable verification of ballots without revealing individual votes.

\section{System Overview}
Consider a simple voting system with 3 candidates. Each voter can allocate up to a maximum of \( \text{max\_credits} = 20 \) credits across the candidates. Let the allocated credits be \( A, B, C \) such that:

\[
A + B + C \leq 20
\]

The goal is to:
\begin{itemize}
    \item Commit to each candidate's vote securely.
    \item Prove the sum of votes does not exceed \( \text{max\_credits} \).
    \item Ensure the proof is zero-knowledge, i.e., no vote information is leaked.
\end{itemize}

\section{Pedersen Commitments}
A Pedersen commitment to a value \( v \) with blinding factor \( r \) is defined as:

\[
\text{Com}(v, r) = g^v h^r
\]

where \( g \) and \( h \) are generators of a prime-order group, and \( r \) is a random scalar ensuring hiding properties.

\subsection{Homomorphic Properties}
Pedersen commitments are additively homomorphic:

\[
\text{Com}(v_1, r_1) \cdot \text{Com}(v_2, r_2) = g^{v_1 + v_2} h^{r_1 + r_2}
\]

This allows commitments to be aggregated without revealing the underlying values.

\section{Ballot Commitment and Proof Generation}
For three candidates, the voter allocates \( A, B, C \) credits. The commitments are:

\[
\text{Com}(A, r_A), \quad \text{Com}(B, r_B), \quad \text{Com}(C, r_C)
\]

The commitment to the total votes \( Z = A + B + C \) is:

\[
\text{Com}(Z, r_Z) = \text{Com}(A, r_A) \cdot \text{Com}(B, r_B) \cdot \text{Com}(C, r_C)
\]

where:

\[
Z = A + B + C, \quad r_Z = r_A + r_B + r_C
\]

\subsection{Proof of Budget Compliance}
To prove \( Z \leq \text{max\_credits} \), we define:

\[
D = \text{max\_credits} - Z
\]

Commit to \( D \) as:

\[
\text{Com}(D, r_D) = g^D h^{r_D}
\]

Using homomorphic properties:

\[
\text{Com}(D, r_D) = \text{Com}(\text{max\_credits}, 0) \cdot \text{Com}(Z, r_Z)^{-1}
\]

Since:

\[
\text{Com}(\text{max\_credits}, 0) = g^{\text{max\_credits}}
\]

We enforce consistency with:

\[
\text{Com}(D, r_D) = g^{\text{max\_credits} - Z} h^{-r_Z}
\]

Hence, \( r_D = -r_Z \) to maintain equality.

\subsection{Range Proof}
A Bulletproof range proof is generated to show:

\[
D \in [0, 255]
\]

This ensures \( Z \leq \text{max\_credits} \) without revealing \( Z \) or \( D \).

\section{Verification Process}
To verify a ballot:

\begin{enumerate}
    \item Check that the sum of commitments equals the commitment to the total:
    \[
    \text{Com}(A) + \text{Com}(B) + \text{Com}(C) = \text{Com}(Z)
    \]
    \item Verify commitment consistency:
    \[
    \text{Com}(D) = \text{Com}(\text{max\_credits}) - \text{Com}(Z)
    \]
    \item Validate the Bulletproof range proof for \( D \).
\end{enumerate}

\section{Example}
Let \( A = 7, B = 8, C = 4 \), and \( \text{max\_credits} = 20 \).

\begin{align*}
Z &= A + B + C = 19 \\
D &= 20 - 19 = 1
\end{align*}

The commitments and range proof are generated accordingly, and verification ensures the correctness without revealing \( A, B, C \).

\section{Conclusion}
This protocol leverages the homomorphic properties of Pedersen commitments and Bulletproof range proofs to securely verify ballots in rated voting systems, ensuring both correctness and privacy.

\end{document}

